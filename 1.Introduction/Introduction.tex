%!TEX root = ../Fast_Contour_Tracing_Algorithm.tex
% -*- root: ../Fast_Contour_Tracing_Algorithm.tex -*-

\section{Introduction}

% A contour is defined as a segment that is one pixel wide and one or more pixels in length, and a boundary is defined as an unbroken contour\cite{Mcqueen2004Contour}. Contours and boundaries are very important information for object representation and image recognition. For example, they are used to separate objects from the background, to calculate the size of an object, to classify a shape, and to find the feature points of an object using the length and shape from its contour pixels\cite{Pratt????Digital,Gose1996Pattern}. Moreover, using the contour information, it is possible to save the shape of objects and restore them to their original shapes for various applications in the field of graphics and vision. Therefore, many researches on contour tracing algorithms for extracting and tracing the contour of an object have been conducted. Most of the algorithms are binary-image-based contour tracing algorithms \cite{Pitas2000Digital,Gose1996Pattern,Das1990Bivariate,Papert1973Uses,Cheong2006Improved,Mirante1982Radial,Pavlidis2012Algorithms}, which trace contours on digitized black-and-white images. Moreover, the focus is on contour tracing algorithms based on color images and gray images []; further, the binary-image-based contour tracing algorithm can be easily applied to color and gray images.

A contour is defined as a segment that is one pixel wide and one or more pixels in length, and a boundary is defined as an unbroken contour \cite{Mcqueen2004Contour}. Contours and boundaries provide very important information for object representation and image recognition. For example, they are used to separate objects from their backgrounds, calculate the sizes of objects, classify shapes, and find the feature points of objects using the length and shape of their contour pixels \cite{Pratt????Digital,Gose1996Pattern}. Moreover, in the field of graphics and vision, it is possible to use the contour information to save the shape of objects and restore them to their original shapes for various applications. Therefore, there have been many studies on contour-tracing algorithms for extracting and tracing the contour of an object. Most of the algorithms are binary-image-based contour-tracing algorithms \cite{Pitas2000Digital,Gose1996Pattern,Das1990Bivariate,Papert1973Uses,Cheong2006Improved,Mirante1982Radial,Pavlidis2012Algorithms}, which trace contours on digitized black-and-white images taken from various image sensors.

% In recent years, as the popularity of camera equipped mobile devices such as mobile phones and personal digital assistants (PDAs) has increased, various real-time applications such as image code recognition, face recognition, optical character recognition (OCR), logo recognition, augmented reality (AR), and mixed reality (MR) have emerged for mobile computing environments [], [], []. Since mobile devices possess restricted hardware resources such as low-performance processors, small-sized memory, low-resolution display, and low battery capacity, that require simpler and faster algorithms for image processing. 

In recent years, with the increasing popularity of smart/wearable mobile sensor devices \cite{Aroca2013Wearable} such as smart phones, smart watch and smart glass, various real-time applications such as image code recognition, face recognition, optical character recognition (OCR), logo recognition, augmented reality (AR), and mixed reality (MR) have emerged for sensor computing\cite{Wakaumi20062D,Brodic2010Basic,Kim2006Rapid,Tian2010RealTime,Zhang2012Robust}. Because smart/wearable mobile sensor devices possess limited hardware resources such as low-performance processors, small-sized memory, low-resolution displays, and low battery capacity, they require simpler and faster algorithms for image processing. 

% Generally, a contour tracing algorithm can be evaluated on the basis of the following four criteria: (1) accuracy of contour tracing, (2) processing time, (3) data size to save the traced contour information, and (4) the ability to accurately restore and enlarge the original contour using the saved data. However, there have been few studies on contour tracing algorithms that have sought to satisfy all these criteria. Some of the conventional algorithms miss contour pixels on specific relative pixel locations, and the others require considerable operation time for tracing the pixels because a shortcut to the local patterns is not considered \cite{Cheong2006Improved,Cheong2006Advanced}. Moreover, most of the algorithms have no data compression ability for saving the contour information, and some of them cannot restore the image perfectly using the saved data \cite{Miyatake1997Contour}. 

Generally, a contour tracing algorithm can be evaluated based on the following four criteria: (1) accuracy of contour tracing, (2) processing time, (3) data size to save the traced contour information, and (4) the ability to accurately restore and enlarge the original contour using the saved data. However, few studies on contour-tracing algorithms have sought to satisfy all of these criteria. Some of the conventional algorithms miss contour pixels that are at specific relative pixel locations, and others require considerable processing time to trace the pixels because shortcuts to the local patterns are not considered \cite{Cheong2006Improved,Cheong2006Advanced}. Moreover, most of the algorithms have no data-compression capabilities that enable them to save the contour information, and some of them cannot restore the image perfectly using the saved data \cite{Miyatake1997Contour}. 

% In this paper, we propose a novel contour tracing algorithm based on pixel following that overcomes the abovementioned problems, i.e., (1) it provides fast and accurate results for contour pixel tracing, (2) contour information can be compressed for reducing the memory size, and (3) it accurately restores the compressed data to the original contour image. In order to achieve the objectives, the proposed algorithm initially distinguishes the local patterns made by adjacent contour pixels, then finds next contour pixel will be traced from the pattern. 

In this paper, we propose a novel contour-tracing algorithm based on pixel following that overcomes the abovementioned problems, i.e., (1) it provides fast and accurate results for contour-pixel tracing, (2) contour information can be compressed to reduce the memory size, and (3) it accurately restores the compressed data to the original contour image. In order to achieve the objectives, the proposed algorithm initially distinguishes the local patterns made by adjacent contour pixels, and it then finds the next contour pixel that will be traced from the pattern. 

% The paper is organized as follows. In the next section, conventional contour tracing algorithms are categorized and their characteristics are introduced. Subsequently, we analyze their performance on the basis of accuracy and speed of the contour tracing process, and then the proposed algorithm is described and its contour tracing procedure, data compression technique, and restoration technique are presented. After that, the comparison of the conventional algorithms and the proposed algorithm on the basis of performance is presented along with experimental results that comprise the number of traced pixels and the processing times for real-time large-sized images. Moreover, the compressed data size and its restored results are compared with the original traced contour pixels. Finally, we summarize the characteristics and experimental results of the proposed algorithm.

The paper is organized as follows. In the next section, we categorize conventional contour-tracing algorithms and introduce their characteristics. Subsequently, we analyze their performance based on the accuracy and speed of the contour-tracing process, and we then describe the proposed algorithm, after which we present its contour-tracing procedure, data-compression technique, and restoration technique. Then,, we present a comparison of the conventional algorithms and the proposed algorithm on the basis of their performance, along with experimental results that include the number of traced pixels and the processing times for real-time large-sized images. Moreover, we compare the compressed data size and its restored results with the original traced contour pixels. Finally, we summarize the characteristics and experimental results of the proposed algorithm.